% Options for packages loaded elsewhere
\PassOptionsToPackage{unicode}{hyperref}
\PassOptionsToPackage{hyphens}{url}
%
\documentclass[
]{article}
\usepackage{amsmath,amssymb}
\usepackage{iftex}
\ifPDFTeX
  \usepackage[T1]{fontenc}
  \usepackage[utf8]{inputenc}
  \usepackage{textcomp} % provide euro and other symbols
\else % if luatex or xetex
  \usepackage{unicode-math} % this also loads fontspec
  \defaultfontfeatures{Scale=MatchLowercase}
  \defaultfontfeatures[\rmfamily]{Ligatures=TeX,Scale=1}
\fi
\usepackage{lmodern}
\ifPDFTeX\else
  % xetex/luatex font selection
\fi
% Use upquote if available, for straight quotes in verbatim environments
\IfFileExists{upquote.sty}{\usepackage{upquote}}{}
\IfFileExists{microtype.sty}{% use microtype if available
  \usepackage[]{microtype}
  \UseMicrotypeSet[protrusion]{basicmath} % disable protrusion for tt fonts
}{}
\makeatletter
\@ifundefined{KOMAClassName}{% if non-KOMA class
  \IfFileExists{parskip.sty}{%
    \usepackage{parskip}
  }{% else
    \setlength{\parindent}{0pt}
    \setlength{\parskip}{6pt plus 2pt minus 1pt}}
}{% if KOMA class
  \KOMAoptions{parskip=half}}
\makeatother
\usepackage{xcolor}
\usepackage[margin=1in]{geometry}
\usepackage{graphicx}
\makeatletter
\def\maxwidth{\ifdim\Gin@nat@width>\linewidth\linewidth\else\Gin@nat@width\fi}
\def\maxheight{\ifdim\Gin@nat@height>\textheight\textheight\else\Gin@nat@height\fi}
\makeatother
% Scale images if necessary, so that they will not overflow the page
% margins by default, and it is still possible to overwrite the defaults
% using explicit options in \includegraphics[width, height, ...]{}
\setkeys{Gin}{width=\maxwidth,height=\maxheight,keepaspectratio}
% Set default figure placement to htbp
\makeatletter
\def\fps@figure{htbp}
\makeatother
\setlength{\emergencystretch}{3em} % prevent overfull lines
\providecommand{\tightlist}{%
  \setlength{\itemsep}{0pt}\setlength{\parskip}{0pt}}
\setcounter{secnumdepth}{-\maxdimen} % remove section numbering
\ifLuaTeX
  \usepackage{selnolig}  % disable illegal ligatures
\fi
\IfFileExists{bookmark.sty}{\usepackage{bookmark}}{\usepackage{hyperref}}
\IfFileExists{xurl.sty}{\usepackage{xurl}}{} % add URL line breaks if available
\urlstyle{same}
\hypersetup{
  hidelinks,
  pdfcreator={LaTeX via pandoc}}

\author{}
\date{\vspace{-2.5em}}

\begin{document}

\#1. Create a data frame using the table below

\#1a.

household\_data \textless- data.frame( Respond\_1 = c(1:20),

Sex =
c(``Female'',``Female'',``Male'',``Female'',``Female'',``Female'',``Female'',``Female'',``Female'',``Female'',``Male'',``Female'',``Female'',``Female'',``Female'',``Female'',``Female'',``Female'',``Male'',``Female''),

FatherOccupation =
c(``Farmer'',``Others'',``Others'',``Others'',``Farmer'',``Driver'',``Others'',``Farmer'',``Farmer'',``Farmer'',``Others'',``Driver'',``Farmer'',``Others'',``Others'',``Farmer'',``Others'',``Farmer'',``Driver'',``Farmer''),

PersonatHome = c(5,7,3,8,5,9,6,7,8,4,7,5,4,7,8,8,3,11,7,6),

Siblingsatschool = c(6,4,4,1,2,1,5,3,1,2,3,2,5,5,2,1,2,5,3,2),

Typeshouse =
c(``Wood'',``Semi-Concrete'',``Concrete'',``Wood'',``Wood'',``Concrete'',``Concrete'',``Wood'',``Semi-Concrete'',``Concrete'',``Semi-Concrete'',``Concrete'',``Semi-Concrete'',``Semi-Concrete'',``Concrete'',``Concrete'',``Concrete'',``Concrete'',``Concrete'',``Semi-Concrete'')

) household\_data

\#1b.

summary(household\_data)

\#1c. mean\_siblings \textless- mean(household\_data\$Siblingsatschool)
is\_mean\_5 \textless- mean\_siblings == 5 print(is\_mean\_5)

\#No because the mean is 2.95

\#1d. first\_two\_rows\_all\_columns \textless- household\_data{[}1:2,
{]} print(first\_two\_rows\_all\_columns)

\#1e. selected\_rows\_columns \textless- household\_data{[}c(3, 5), c(2,
4){]} print(selected\_rows\_columns)

\#1f. types\_houses \textless- household\_data\$Typeshouse

\#1g. male\_farmers \textless-
household\_data{[}household\_data\(Sex == "Male" & household_data\)FatherOccupation
== ``Farmer'', {]} print(male\_farmers)

\#1h.

female\_greater\_than\_5\_siblings \textless-
household\_data{[}household\_data\(Sex == "Female" & household_data\)Siblingsatschool
\textgreater= 5, {]} print(female\_greater\_than\_5\_siblings)

\#2 df \textless- data.frame( Ints = integer(0), Doubles = double(0),
Characters = character(0), Logicals = logical(0), Factors = factor(NA,
levels = character(0)), stringsAsFactors = FALSE )

print(``Structure of the empty dataframe:'') str(df)

\#2a. The data frame has no data.

\#3 household\_data \textless- data.frame( Respondents = c(1:10), Sex =
c(``Male'', ``Female'', ``Female'', ``Male'', ``Male'', ``Female'',
``Female'', ``Male'', ``Female'', ``Male''), FatherOccupation = c(1, 2,
3, 3, 1, 2, 2, 3, 1, 3), PersonatHome = c(5, 7, 3, 8, 6, 4, 4, 2, 11,
6), Siblingsatschool = c(2, 3, 0, 5, 2, 3, 1, 2, 6, 2), Typeshouse =
c(``Wood'', ``Congrete'', ``Congrete'', ``Wood'', ``Semi-Congrete'',
``Semi-Congrete'', ``Wood'', ``Semi-Congrete'', ``Semi-Congrete'',
``Congrete'') ) household\_data

\#3a. write.csv(household\_data, file = ``HouseholdData.csv'', row.names
= FALSE)

\#3b. imported\_data \textless- read.csv(``HouseholdData.csv'')

\#3b.(2) imported\_data\(Sex <- factor(imported_data\)Sex, levels =
c(``Male'', ``Female''))
imported\_data\(Sex <- as.integer(imported_data\)Sex)

\#3c. imported\_data\(Typeshouse <- factor(imported_data\)Typeshouse,
levels = c(``Wood'', ``Concrete'', ``Semi-Concrete''))
imported\_data\(Typeshouse <- as.integer(imported_data\)Typeshouse)

\#3d.
imported\_data\(FatherOccupation <- factor(imported_data\)FatherOccupation,
levels = c(``Farmer'', ``Driver'', ``Others''))
imported\_data\(FatherOccupation <- as.integer(imported_data\)FatherOccupation)

\#3e. female\_drivers \textless-
imported\_data{[}imported\_data\(Sex == 2 & imported_data\)FatherOccupation
== 2, {]} print(female\_drivers)

\#3f. greater\_than\_5\_siblings \textless-
imported\_data{[}imported\_data\$Siblingsatschool \textgreater= 5, {]}
print(greater\_than\_5\_siblings)

\#4. Interpret the graph Figure 3's graph illustrates how daily
attitudes of people affect our world in significant ways. In other
words, we always allow ourselves to express our thoughts and the
knowledge we acquire on a daily basis.

\end{document}
